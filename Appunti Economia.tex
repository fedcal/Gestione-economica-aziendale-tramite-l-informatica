\documentclass[a4paper]{extarticle}
\usepackage[utf8]{inputenc}
\usepackage{hyperref}
\usepackage{geometry}
\usepackage{fancyhdr}
\usepackage{graphicx} % libreria per le immagini
\usepackage{amssymb} %libreria per i simboli (ex. alfabeto reco)
\usepackage{algorithm2e} %libreria per scrivere pseudocodice
\usepackage{longtable}
\usepackage{caption}
\usepackage{lastpage}
\usepackage{adjustbox}
\usepackage{ellipsis}
\usepackage{listings}
\usepackage{xcolor}
\usepackage{mathtools}

\definecolor{backcolour}{rgb}{0.95,0.95,0.96}
\lstdefinestyle{mystyle}{
	backgroundcolor=\color{backcolour},
	numbers=left,
	numbersep=5pt,
}
\lstset{style=mystyle, escapeinside={(*}{*)}}

\setlength{\parindent}{0em}%indentazione paragrafo
\setlength{\parskip}{1em}%spazio tra paragrafi
\renewcommand{\baselinestretch}{1.3}%interlinea
\graphicspath{ {./} }
\geometry{
    a4paper,
    left=10mm,
    right=10mm,
    bottom=20mm
}


\hypersetup{
    colorlinks=true,
    linkcolor=blue,
    filecolor=blue,      
    urlcolor=blue,
    pdftitle={Overleaf Example},
    pdfpagemode=FullScreen
}

\pagestyle{fancy}
\fancyhf{}
\rhead{Federico Calò}
\lhead{Appunti economia}
\cfoot{  \thepage }


\title{Appunti economia}
\author{\href{http://www.federicocalo.dev}{Federico Calò} }
\date{}

\begin{document}
\maketitle
\newpage
\tableofcontents
\voffset -30pt

\newpage

\section*{Premessa}

Questi appunti hanno lo scopo specifico di creare una conoscenza nel dominio economico per chi volesse affacciarsi a questa materia. Io sono Federico Calò, sviluppatore software con un diploma di tecnico commerciale. Sono cresciuto immerso nell'economia, in quanto mio padre ha condotto studi in questa materia e lavora nell'ambito bancario. Il mio scopo attuale è quello di unire l'ambito bancario a quello informatico, creando gestionali che supportano la gestione aziendale, dagli organigrammi alla creazione del bilancio dell'esercizio, organizzando in maniera corretta i dati all'interno dei database.


\end{document}



